%%%%%%%%%%%%%%%%%%%%%%%%%%%%%%%%%%%%%%%%%
% 
% K-CV -- Klenske Curriculum Vitae
% 
% Simplified Friggeri CV 
% by Edgar Klenske (ed.klenske@gmx.de)
% https://github.com/eklenske/CV
%
% Forked from:
% Jelmer Tiente 
% https://github.com/JelmerT/CV
%
% Based on original work by:
% Adrien Friggeri (adrien@friggeri.net)
% https://github.com/afriggeri/CV
%
% License:
% CC BY-NC-SA 3.0 (http://creativecommons.org/licenses/by-nc-sa/3.0/)
%
%%%%%%%%%%%%%%%%%%%%%%%%%%%%%%%%%%%%%%%%%

\documentclass[]{k-cv} % Add 'print' as an option into the square
                       % bracket to remove colors from this template
                       % for printing

\usepackage{marginnote}
\usepackage{hyperref}
\begin{document}
\header{Ekaterina }{ILIN}{PhD student in astrophysics} % Your name and current job

%-------------------------------------------------------------------------------
%	SIDEBAR SECTION
%-------------------------------------------------------------------------------

\begin{aside} % In the aside, each new line forces a line break
\section{date of birth}
\color{gray}1992-09-11
\section{contact}
Carl-D\"ahne-Str. 10
14469 Potsdam
Germany
~
+49 (176) 531 351 97
+49 (331) 749 928 0
~
\href{mailto:eilin@aip.de}{eilin@aip.de}
\href{https://github.com/ekaterinailin}{github/ekaterinailin}
%\href{http://facebook.com/johnsmith}{fb://jsmith}
\section{languages}
Russian, German (bilingual)
English C1
French B2
\section{skills}
Python, Unix, LaTeX, Git(Hub)
\end{aside}

%-------------------------------------------------------------------------------
%	EDUCATION SECTION
%----------------------------------------------------------------------------------------

\section{education}

\begin{entrylist}
\entry
{2018 \to \textbf{2022}}
{PhD {\normalfont in Astrophysics}}
{Leibniz Institute for Astrophysics Potsdam (AIP), Germany}
{Supervisor: Prof. Dr. Katja Poppenh\"ager\\
\emph{Stellar Flares in Star-Exoplanet Systems and Young Stars} \vspace{0.2cm}\\ I am interested in magnetic star-planet interactions, and their long-term effects on exoplanet habitability and stellar evolution.}
%------------------------------------------------
%------------------------------------------------
\entry
{2016 \to 2018}
{Master {\normalfont of Science in Astrophysics}}
{University of Potsdam, Germany}
{Supervisor: Prof. Dr. Klaus Strassmeier\\
\emph{Flares in Clusters using K2 Data} \\ 
\emph{1,0 "with distinction"}\vspace{0.2cm}\\
This thesis quantified the relation between flaring activity on cool stars and their masses and ages for ZAMS to 600 Myr and K to mid-M dwarfs.}
%------------------------------------------------


%------------------------------------------------
\entry
{2012 \to 2016}
{Bachelor {\normalfont of Science in Physics}}
{Karlsruhe Institute for Technology, Germany}
{Specialization in Biophysics\\
Supervisor: Prof. Dr. Ulrich Nienhaus\\
\emph{Uptake of TAT-coated Gold Nanoclusters in Live HeLa-Cells}\\
\emph{1,4 "very good"}}
%------------------------------------------------
\end{entrylist}

\section{work and teaching}

\begin{entrylist}

%------------------------------------------------
\entry
{2018}
{Leibniz Institute for Astrophysics (AIP)}
{Potsdam, Germany}
{\emph{Research Assistant}}
%------------------------------------------------
\entry
{2016 \to 2018}
{University of Potsdam}
{Potsdam, Germany}
{\emph{Teaching assistant} for undergraduates.}
%------------------------------------------------
\entry
{2011 \to 2012}
{SchulLV GmbH}
{Karlsruhe, Germany}
{\emph{Author} of teaching aid in mathematics.}
%------------------------------------------------
\end{entrylist}





\section{presentations}

\begin{entrylist}
%------------------------------------------------
\entry
{2019}
{Conference talk}
{16th Potsdam Thinkshop: \vspace{-.1cm}\\\null\hfill
The rotation periods of cool stars:\vspace{-.1cm}\\\null\hfill Measurements, uses, connections and prospects \vspace{-.1cm}\\\null\hfill
Potsdam, Germany}
{At a gallop: stellar flares from Kepler/K2 and TESS}
%------------------------------------------------
%\entry
%{2019}
%{Poster}
%{3rd Advanced School on Exoplanetary Demographics\vspace{-.1cm}\\\null\hfill Vietri sul Mare, Italy}
%{The frequency of super-Carrington flares in the solar-age cluster M67}
%------------------------------------------------
\entry
{2019}
{Conference talk}
{Planet-Star Connections in the Era of TESS and Gaia\vspace{-.1cm}\\\null\hfill Santa Barbara, CA, USA}
{Stellar magnetic evolution: flaring activity in K2 open clusters}
%------------------------------------------------
%\entry
%{2019}
%{Seminar talk}
%{Centre for Star and Planet Formation (STARPLAN)\vspace{-.1cm}\\\null\hfill Copenhagen, Denmark}
%{Stellar magnetic evolution: superflares}
%%------------------------------------------------
\entry
{2018}
{Conference poster}
{CoolStars20\vspace{-.1cm}\\\null\hfill Boston, MA, USA}
{Exploring flaring activity as an age indicator using open cluster data}
%------------------------------------------------
\entry
{2018}
{Workshop talk}
{Dwarf Stars and Clusters with K2\vspace{-.1cm}\\\null\hfill Boston, MA, USA}
{In search of the flaring-age relation.}
%------------------------------------------------
\end{entrylist}


%-------------------------------------------------------------------------------
%	AWARDS SECTION
%-------------------------------------------------------------------------------
%
%\clearpage
%\smallheader{Ekaterina}{Ilin}

\section{awards and scholarships}

\begin{entrylist}
%------------------------------------------------
\entry
{2020 \to ongoing}
{Doctoral Scholarship}
{German Academic Scholaship Foundation}
{Includes a stipend, and additional academic and financial support throughout the doctoral studies.}

\entry
{2019}
{Graduation Award}
{Physikalische Gesellschaft zu Berlin, Berlin, Germany}
{Award tendered to excellent master theses earned at universities in Berlin and Potsdam.}


\entry
{2018}
{Visiting Scientist}
{NASA Ames Research Center, Mountain View, CA, USA}
{Awarded funding for a research visit at the Kepler mission headquarters.}

\entry
{2018}
{Hackathon Challenge Winner}
{HackHPI, Hasso Plattner Institute, Potsdam, Germany}
{Data science challenge by \emph{Tableau}.}

\entry
{2014}
{Erasmus+ Scholarship}
{Universitet i Oslo, Norway}
{Awarded funding for six month of undergraduate studies.\\
Focus: Space Physics, Computational Physics}

\entry
{2012 \to 2018}
{Full Scholarship}
{German Academic Scholarship Foundation}
{Includes a stipend, and additional academic and financial support throughout the undergraduate and graduate studies.}

%------------------------------------------------
\end{entrylist}




%-------------------------------------------------------------------------------
%	PUBLICATIONS SECTION
%-------------------------------------------------------------------------------

\section{refereed publications}
%\reversemarginpar
%\marginnote{\color{gray}{refereed}}
\bibentry{Flares in open clusters with K2. I. M45 (Pleiades), M44 (Praesepe) and M67}{Ilin, Ekaterina; Schmidt, Sarah J.; Davenport, James R. A.; Strassmeier, Klaus G.}{2019, Astronomy \& Astrophysics, 622, A133/16.\\ \texttt{\href{https://doi.org/10.1051/0004-6361/201834400}{doi:10.1051/0004-6361/201834400
}, \href{https://arxiv.org/abs/1812.06725}{ArXiv:1812.06725}}}

%\clearpage
%\smallheader{Ekaterina}{Ilin}

\section{volunteering}
\begin{entrylist}
\entry
{2017 \to ongoing}
{Python in astronomy}
{University of Potsdam\vspace{-.1cm}\\\null\hfill Leibniz Institute for Astrophysics Potsdam - AIP}
{Running a Python code review group with the master students in astrophysics on a biweekly basis in 2017/2018. Running the Python Club at AIP since 2019.\\}
%------------------------------------------------
\entry
{2016 \to ongoing}
{Effective Altruism}
{Effective Altruism Berlin, Potsdam\vspace{-.1cm}
\\\null\hfill Studienstiftung d. dt. Volkes
\vspace{-.1cm}\\\null\hfill Netzwerk f\"ur Effektiven Altruismus Deutschland e.V.}
{Organising interdisciplinary workshops for 20-120 participants, reading groups, and recreational activities; local group head in Potsdam. Currently member of the excutive board of the German Network for Effective Altruism (NEAD e.V.).\\}
%------------------------------------------------
\entry
{2013 \to ongoing}
{Junior scientists network}
{Deutsches Jungforschernetzwerk - juFORUM e.V.}
{Organising interdisciplinary workshops and symposia for students and young researchers in STEM, presenting at science fairs, publishing articles in the club magazine; member of the executive board.}
%------------------------------------------------
\end{entrylist}
\end{document}
