%%%%%%%%%%%%%%%%%%%%%%%%%%%%%%%%%%%%%%%%%
% 
% K-CV -- Klenske Curriculum Vitae
% 
% Simplified Friggeri CV 
% by Edgar Klenske (ed.klenske@gmx.de)
% https://github.com/eklenske/CV
%
% Forked from:
% Jelmer Tiente 
% https://github.com/JelmerT/CV
%
% Based on original work by:
% Adrien Friggeri (adrien@friggeri.net)
% https://github.com/afriggeri/CV
%
% License:
% CC BY-NC-SA 3.0 (http://creativecommons.org/licenses/by-nc-sa/3.0/)
%
%%%%%%%%%%%%%%%%%%%%%%%%%%%%%%%%%%%%%%%%%

\documentclass[9.5pt]{k-cv} % Add 'print' as an option into the square
                       % bracket to remove colors from this template
                       % for printing

\usepackage{marginnote}
\usepackage{hyperref}
\begin{document}
\header{Ekaterina }{ILIN}{PhD student in astrophysics} % Your name and current job

%-------------------------------------------------------------------------------
%	SIDEBAR SECTION
%-------------------------------------------------------------------------------
% In the aside, each new line forces a line break
%\section{date of birth}
%1992-09-11
% \begin{aside} 
% \color{gray}
% \section{contact}
% Carl-D\"ahne-Str. 10
% 14469 Potsdam
% Germany
% ~
% +49 (176) 531 351 97
% %+49 (331) 749 928 0
% ~
% \href{mailto:eilin@aip.de}{eilin@aip.de}
% \href{https://github.com/ekaterinailin}{github/ekaterinailin}
% %\href{http://facebook.com/johnsmith}{fb://jsmith}
% \section{languages}
% Russian, German (bilingual)
% English C2
% French B2
% \section{skills}
% Python, Unix, LaTeX, Git(Hub)
% \end{aside}

%-------------------------------------------------------------------------------
%	EDUCATION SECTION
%----------------------------------------------------------------------------------------


\section{education}

\begin{entrylist}
\entry
{2018 \to \textbf{07/2022}}
{PhD {\normalfont in Astrophysics} @ Leibniz Institute for Astrophysics Potsdam (AIP), Germany}
{}
{Supervisor: Prof. Dr. Katja Poppenh\"ager\\
\emph{Stellar flares in star-exoplanet systems and young stars}\vspace{.3cm}\\
$\rightarrow$ I am interested in the magnetism of small stars, its origins, long-term evolution, and the space weather it creates for (exo)planets.\vspace{.3cm}}

%------------------------------------------------
%------------------------------------------------
\entry
{2016 \to 2018}
{Master {\normalfont of Science in Astrophysics} @ University of Potsdam, Germany}
{}
{Supervisor: Prof. Dr. Klaus Strassmeier, mentor: Dr. Sarah J. Schmidt\\
\emph{Flares in clusters using K2 data}\\ \emph{1,0 "with distinction"}}%\\
%This thesis quantified the relation between flaring activity on cool stars and their masses and ages for ZAMS to 600 Myr and K to mid-M dwarfs.}
%------------------------------------------------


%------------------------------------------------
\entry
{2012 \to 2016}
{Bachelor {\normalfont of Science in Physics} @ Karlsruhe Institute for Technology, Germany}
{}
{Specialization in biophysics, Supervisor: Prof. Dr. Ulrich Nienhaus\\
\emph{Uptake of TAT-coated gold nanoclusters in live HeLa cells}\\
\emph{1,4 "very good"}}
%------------------------------------------------
\end{entrylist}


%-------------------------------------------------------------------------------
%	PUBLICATIONS SECTION
%-------------------------------------------------------------------------------

\section{refereed publications}
%\reversemarginpar
%\marginnote{\color{gray}{refereed}}

\bibentry{Localizing flares to understand stellar magnetic fields and space weather in exo-systems }{Ilin, E.; Poppenhäger, K.; Alvarado-G\'omez, J. D}{2021, Astronomische Nachrichten,  \texttt{\href{https://doi.org/10.1002/asna.20210111}{doi:10.1002/asna.20210111} \href{https://arxiv.org/abs/2112.09676}{arXiv:2112.09676}}}

\bibentry{Giant white-light flares on fully convective stars occur at high latitudes}{Ilin, E.; Poppenhäger, K.; Schmidt, S. J.; J\"arvinen, S. P.; Newton, E. R.; Alvarado-G\'omez, J. D; Pineda, S. J.; Davenport, J. R. A.; Oshagh, M.; Ilyin, I.}{2021, Monthly Notices of the Royal Astronomical Society, 507/2, pp 1723–1745  \texttt{\href{https://doi.org/10.1093/mnras/stab2159}{doi:10.1093/mnras/stab2159} \href{https://arxiv.org/abs/2108.01917}{arXiv:2108.01917}}}


\bibentry{AltaiPony -- Flare science in Kepler, K2 and TESS light curves}{Ilin, E.}{2021, Journal of Open Source Software, 6(62), 2845 \texttt{\href{https://doi.org/10.21105/joss.02845}{10.21105/joss.02845}}}

\bibentry{Flares in Open Clusters with K2. II. Pleiades, Hyades, Praesepe, Ruprecht 147, and M67}{Ilin, E.; Schmidt, S. J.; Poppenhäger, K.; Davenport, J. R. A.; Kristiansen, M. H.; Omohundro, M.}{2021, Astronomy \& Astrophysics, 645, A42, 25 pp. \texttt{\href{https://doi.org/10.1051/0004-6361/202039198}{doi:10.1051/0004-6361/202039198} \href{https://arxiv.org/abs/2010.05576}{arXiv:2010.05576}}}

\bibentry{Flares in Open Clusters with K2. I. M45 (Pleiades), M44 (Praesepe) and M67}{Ilin, E.; Schmidt, S. J.; Davenport, J. R. A.; Strassmeier, K. G.}{2019, Astronomy \& Astrophysics, 622, A133/16. \texttt{\href{https://doi.org/10.1051/0004-6361/201834400}{doi:10.1051/0004-6361/201834400
} \href{https://arxiv.org/abs/1812.06725}{arXiv:1812.06725}}}
\section{teaching and supervision}

\begin{entrylist}


%------------------------------------------------
\entry
{2020 \to ongoing}
{Mentoring graduate student @ University of Heidelberg}
{}
{Mentoring Aaron Maas during his first peer-reviewed research project on TRAPPIST-1 flares}
%, during his first peer-reviewed research project on TRAPPIST-1 flares with MUSCAT.
%------------------------------------------------
\entry
{2020}
{Teaching assistant @ University of Potsdam }
{}
{Galaxies and Cosmology undergraduate course}
%------------------------------------------------
% \entry
% {2018}
% {Research Assistant @ Leibniz Institute for Astrophysics (AIP) \vspace{-.6cm}}
% {}
% {}
%------------------------------------------------
\entry
{2016 \to 2018}
{Teaching assistant @ University of Potsdam }
{}
{Introduction to mathematical methods in physics}
%------------------------------------------------
%\entry
%{2011 \to 2012}
%{SchulLV GmbH}
%{Karlsruhe, Germany}
%{\emph{Author} of teaching aid in mathematics.}
%%------------------------------------------------
\end{entrylist}
\newpage
\section{telescope time}

\begin{entrylist}

\entry
{2021}
{XMM-Newton, 36ks priority A}
{}
{PI: \textbf{E. Ilin}, Co-Is: K. Poppenh\"ager, Co-Is: , B. Stelzer}

\entry
{2020 }
{SALT HRS, 2.8 hours DDT}
{}
{PI: K. Poppenh\"ager, Co-Is: \textbf{E. Ilin}, S. J. Schmidt}




\end{entrylist}

\section{awards, scholarships, funding}
\begin{entrylist}
%------------------------------------------------
\entry
{2021 \to ongoing}
{Fulbright Scholarship @ American Museum of Natural History}
{ }
{Awarded to doctoral students by the German-American Fulbright Commission. Includes a stipend, travel grant, and academic support throughout a  six month research stay in the USA.}

\entry
{2020 \to ongoing}
{Doctoral Scholarship @ German Academic Scholarship Foundation}
{}
{Includes a stipend, travel grants, and academic support throughout the doctoral studies.}

\entry
{2019}
{Graduation Award @ Physikalische Gesellschaft zu Berlin, Berlin, Germany}
{}
{Award for excellent master theses in physics earned at universities in Berlin and Potsdam.}


\entry
{2018}
{Visiting Scientist @ NASA Ames Research Center, Mountain View, CA, USA}
{}
{Awarded funding for a research visit at the Kepler mission headquarters.}

%\entry
%{2018}
%{Hackathon Challenge Winner}
%{HackHPI, Hasso Plattner Institute, Potsdam, Germany}
%{Data science challenge by \emph{Tableau}.}

\entry
{2014}
{Erasmus+ Scholarship @ Universitet i Oslo, Norway}
{}
{Awarded funding for six months of undergraduate studies.
Focus: space physics, computational physics}

\entry
{2012 \to 2018}
{Full Scholarship @ German Academic Scholarship Foundation}
{}
{Includes a stipend, travel grants, and academic support throughout the undergraduate and graduate studies.}

%------------------------------------------------
\end{entrylist}








\section{academic presentations and posters (selection)}

\begin{entrylist}
%------------------------------------------------
\entry
{2021}
{Invited talk @ Harvard Smithsonian Center for Astrophysics Exoplanet Pizza Lunch}
{}%{\\ virtual}
{Magnetic worlds: What flares tell us about small stars and their planets}

\entry
{2021}
{Contributed talk @ European Astronomical Society Meeting}
{}%{\\ virtual}

{The coolest stars may not hit their planets with their flaring all that much}
%------------------------------------------------
\entry
{2021}
{Contributed talk @ XMM-Newton Science Workshop}
{}%{\\ virtual}
{Giant white-light flares on fully convective stars occur at high latitudes}
%------------------------------------------------
%\entry
%{2021}
%{Virtual Poster @ CoolStars20.5}
%{Cambridge Workshops of Cool Stars, Stellar Systems and the Sun \vspace{-.1cm}\\\null\hfill  CoolStars 20.5\vspace{-.1cm}\\\null\hfill
%virtual, global}
%{\href{https://zenodo.org/record/4558792}{Giant white-light flares on fully convective stars occur at high latitudes}}
%------------------------------------------------
\entry
{2020}
{Contributed talk @ Annual Meeting of the German Astronomical Society}
{}%{\\ virtual}
{Flares, mass, age, AND rotation. Calibrating the flaring-age-mass relation in open clusters}
%------------------------------------------------
%\entry
%{2020}
%{Poster exhibition}
%{Royal Astronomical Society\vspace{-.1cm}\\\null\hfill
%Early Career Poster Exhibition\vspace{-.1cm}\\\null\hfill
%virtual, global}
%{Where do giant flares occur on small stars?}
%------------------------------------------------
\entry
{2019}
{Contributed talk @ \href{https://thinkshop.aip.de/16/cms/} {16th Potsdam Thinkshop}}
{}%{\\AIP, Potsdam, Germany}

{At a gallop: stellar flares from Kepler/K2 and TESS}
%------------------------------------------------
%\entry
%{2019}
%{Poster}
%{3rd Advanced School on Exoplanetary Demographics\vspace{-.1cm}\\\null\hfill Vietri sul Mare, Italy}
%{The frequency of super-Carrington flares in the solar-age cluster M67}
%------------------------------------------------
\entry
{2019}
{Contributed talk @ \href{https://www.kitp.ucsb.edu/activities/exostar-c19}{Planet-Star Connections in the Era of TESS and Gaia}}
{}%{\\Santa Barbara, CA, USA}

{Stellar magnetic evolution: flaring activity in K2 open clusters}
%------------------------------------------------
\entry
{2019}
{Seminar talk @ Centre for Star and Planet Formation (STARPLAN)}
{}%{\\ Copenhagen, Denmark}
{Stellar magnetic evolution: superflares}
% %%------------------------------------------------
\entry
{2018}
{Conference poster @ \href{http://coolstars20.cfa.harvard.edu/}{CoolStars20}}
{}% {\\ Boston, MA, USA}
{Exploring flaring activity as an age indicator using open cluster data}
%------------------------------------------------
\entry
{2018}
{Contributed talk @ \href{https://keplergo.github.io/KeplerScienceWebsite/workshop-on-dwarf-stars-and-clusters-with-k2-registration-now-open-space-is-limited.html}{Dwarf Stars and Clusters with K2}}
{}% {\\Boston, MA, USA}
{In search of the flaring-age relation.}
%------------------------------------------------
\end{entrylist}


%-------------------------------------------------------------------------------
%	AWARDS SECTION
%-------------------------------------------------------------------------------
%
%\clearpage
%\smallheader{Ekaterina}{Ilin}

%\clearpage
%\smallheader{Ekaterina}{Ilin}
\newpage
\section{outreach and civic involvement}
\begin{entrylist}
%\entry
%{2017 \to ongoing}
%{Python in astronomy}
%{University of Potsdam\vspace{-.1cm}\\\null\hfill Leibniz Institute for Astrophysics Potsdam - AIP}
%{Running a Python code review group with the master students in astrophysics on a biweekly basis in 2017/2018. Running the Python Club at AIP since 2019.\\}
%%------------------------------------------------
\entry
{2021 \to ongoing}
{Student representative @ Internal Scientific Committee at AIP}
{}
{Duties include: \textbf{representing (under)graduate and PhD students} affiliated with the AIP; running social events for the institute staff; on-boarding of new students to AIP}

\entry
{2021}
{Public talks @ YouTube channel "Sterne, Weltall und das Leben"}
{}
{\texttt{\href{https://www.youtube.com/watch?v=LLHLobUifeY}{30-min YouTube video}} and \texttt{\href{https://www.youtube.com/watch?v=YRw_tIpspRw}{follow-up 12-min YouTube video}} about electromagnetic star-planet interactions on the \textbf{popular German astronomy channel} "Sterne, Weltall und das Leben", $>$ 37k views total.}

\entry
{2019}
{Interview @ Women in Research Blog}
{}
{\texttt{\href{https://womeninresearchblog.wordpress.com/2019/06/26/ekaterina-germany/}{Blog post}} about my understanding of scientific research and opinions on how to support \textbf{women in STEM}.}

%%------------------------------------------------

\end{entrylist}

\textcolor{red}{ \rule{19cm}{1pt} }
 
\begin{entrylist}

\entry
{\Large{\textit{\textbf{languages}}}}
{Russian, German (bilingual), English (C2), French (B2)}
{}{}

\entry
{\Large{\textit{\textbf{skills}}}}
{Python, Unix, LaTeX, Git(Hub)}
{}{}

\entry
{\Large{\textit{\textbf{contact}}}}
{Carl-D\"ahne-Str. 10, 14469 Potsdam, Germany,\\~+49 (176) 531 351 97, \href{mailto:eilin@aip.de}{eilin@aip.de}, \href{https://github.com/ekaterinailin}{github/ekaterinailin}}
{}{}
\end{entrylist}

\end{document}
